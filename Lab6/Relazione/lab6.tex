\documentclass[12pt]{article}
\usepackage[english]{babel}
\usepackage{natbib}
\usepackage{url}
\usepackage[utf8x]{inputenc}
\usepackage{amsmath}
\usepackage{graphicx}
\graphicspath{{images/}}
\usepackage{parskip}
\usepackage{fancyhdr}
\usepackage{vmargin}
\usepackage{xcolor}
\usepackage{siunitx}
\usepackage{physics}
\setmarginsrb{3 cm}{2 cm}{3 cm}{2 cm}{1 cm}{1.5 cm}{1 cm}{1.5 cm}

\title{Lab 06}													% Title
\author{G 03}														% Author
\date{14 may 2019}														% Date

\makeatletter
\let\thetitle\@title
\let\theauthor\@author
\let\thedate\@date
\makeatother

\pagestyle{fancy}
\fancyhf{}
\rhead{\theauthor}
\lhead{\thetitle}
\cfoot{\thepage}
\newcommand{\mis}[3]{(#1 \pm #2) \ #3}
\newcommand{\misp}[3]{(#1 \#3 \pm #2}
\begin{document}

%%%%%%%%%%%%%%%%%%%%%%%%%%%%%%%%%%%%%%%%%%%%%%%%%%%%%%%%%%%%%%%%%%%%%%%%%%%%%%%%%%%%%%%%%

\begin{titlepage}
	\centering
    \vspace*{0.5 cm}
    \includegraphics[scale = 0.75]{polito.jpg}\\[1.0 cm]				% University Logo
    \textsc{\LARGE Politecnico di Torino}\\[2.0 cm]						% University Name
	\textsc{\Large Digital systems electronics\\ A.A. 2018/2019}\\[0.5 cm]		% Course Code
	\textsc{\Large Prof. G. Masera}\\[0.5 cm]		% Nome del Professore
	\rule{\linewidth}{0.2 mm} \\[0.4 cm]
	{ \huge \bfseries \thetitle \\ \small \thedate}\\
	\rule{\linewidth}{0.2 mm} \\[1.5 cm]
	
	\begin{minipage}{0.4\textwidth}
		\begin{flushleft} \large
			Berchialla Luca\\												%Cognomi e nomi
			Laurasi Gjergji
			\\
			
			Mattei Andrea\\
            Lombardo Domenico Maria\\
            
			\end{flushleft}
			\end{minipage}~
			\begin{minipage}{0.4\textwidth}
            
			\begin{flushright} \large
			236032\\													%Matricole
			238259\\
            233755\\
            233959\\
            
		\end{flushright}
        
	\end{minipage}\\[2 cm]
	
\end{titlepage}

%%%%%%%%%%%%%%%%%%%%%%%%%%%%%%%%%%%%%%%%%%%%%%%%%%%%%%%%%%%%%%%%%%%%%%%%%%%%%%%%%%%%%%%%%
\newpage

\section*{Introduction to the assigned problem}
This relation deals with the design of a simple digital filter. From the given functional specifications, a final digital circuit has been implemented in VHDL including memories, control unit and data-path. 
\newline The design process will be analyzed using  a bottom-up approach, starting from the individual components and their interconnections to the final custom designed FSM.





\section*{Overall specs description}

As already mentioned, the final  purpose of this activity is to implement a digital filter following the relation:
\[Y(n) = -0.5X(n) - 2X(n-1) + 4X(n-2) + 0.25X(n-3) \]

where $ X(n) $ are the input stream data and $ Y(n) $ the corresponding filtered data generated by the top equation.

The circuit starts by means of a $START$ signal which enable a loading process of the input data into a 1 kByte memory. Then, the circuit automatically filters the data stream exploiting the equation already provided and loads the output filtered data into a second memory. Finally, the circuit reports end of process asserting an $HIGH$ logical value to the $DONE$ signal and awaits until the next $START$.

\subsection*{Shift registers}
\subsection*{Memories}
\subsection*{Adder}
\subsection*{Counter}


%\begin{figure}[h]
%	\centering
%	\includegraphics[scale = 0.47]{immagini/Berchialla/STG.jpg}
%	\caption{State diagram of the FSM}
%\end{figure}
\end{document}




