\documentclass[12pt]{article}
\usepackage[english]{babel}
\usepackage{natbib}
\usepackage{url}
\usepackage[utf8x]{inputenc}
\usepackage{amsmath}
\usepackage{graphicx}
\graphicspath{{images/}}
\usepackage{parskip}
\usepackage{fancyhdr}
\usepackage{vmargin}
\usepackage{xcolor}
\usepackage{siunitx}
\usepackage{physics}
\setmarginsrb{3 cm}{2 cm}{3 cm}{2 cm}{1 cm}{1.5 cm}{1 cm}{1.5 cm}

\title{Lab 02}													% Title
\author{G 03}														% Author
\date{24 Mar 2019}														% Date

\makeatletter
\let\thetitle\@title
\let\theauthor\@author
\let\thedate\@date
\makeatother

\pagestyle{fancy}
\fancyhf{}
\rhead{\theauthor}
\lhead{\thetitle}
\cfoot{\thepage}
\newcommand{\mis}[3]{(#1 \pm #2) \ #3}
\newcommand{\misp}[3]{(#1 \#3 \pm #2}
\begin{document}

%%%%%%%%%%%%%%%%%%%%%%%%%%%%%%%%%%%%%%%%%%%%%%%%%%%%%%%%%%%%%%%%%%%%%%%%%%%%%%%%%%%%%%%%%

\begin{titlepage}
	\centering
    \vspace*{0.5 cm}
    \includegraphics[scale = 0.75]{polito.jpg}\\[1.0 cm]				% University Logo
    \textsc{\LARGE Politecnico di Torino}\\[2.0 cm]						% University Name
	\textsc{\Large Digital systems electronics\\ A.A. 2018/2019}\\[0.5 cm]		% Course Code
	\textsc{\Large Prof. G. Masera}\\[0.5 cm]		% Nome del Professore
	\rule{\linewidth}{0.2 mm} \\[0.4 cm]
	{ \huge \bfseries \thetitle \\ \small \thedate}\\
	\rule{\linewidth}{0.2 mm} \\[1.5 cm]
	
	\begin{minipage}{0.4\textwidth}
		\begin{flushleft} \large
			Berchialla Luca\\												%Cognomi e nomi
			Laurasi Gjergji
			\\
			
			Mattei Andrea\\
            Lombardo Domenico Maria\\
            
			\end{flushleft}
			\end{minipage}~
			\begin{minipage}{0.4\textwidth}
            
			\begin{flushright} \large
			236032\\													%Matricole
			238259\\
            233755\\
            233959\\
            
		\end{flushright}
        
	\end{minipage}\\[2 cm]
	
\end{titlepage}

%%%%%%%%%%%%%%%%%%%%%%%%%%%%%%%%%%%%%%%%%%%%%%%%%%%%%%%%%%%%%%%%%%%%%%%%%%%%%%%%%%%%%%%%%

\section{ Controlling a 7-segments display}

	Figure 1 shows a 7-segment decoder module whose input bits $C2 C1 C0$ drive a 7 segment display through the bits $HEX0_{0}->HEX0_{6}$ 
	
	\begin{figure}[h]
		\centering
		\includegraphics[scale = 0.7]{Niki_puntoA/system.png}
		\caption{7-segment decoder + display}
	\end{figure}	

	Figure 2 shows the truth table to be implemented for the 7-segment decoder. As shown just the characters $H E L O$ will be implemented.
	The sequent logical states can be easily derived from the table:
	\[HEX6=\overline{C2}\cdot\overline{C1}\]
	\[HEX5=\overline{C2}\]
	\[HEX4=\overline{C2}\]
	\[HEX3=\overline{C2}\cdot(C0+C1)\]
	\[HEX2=\overline{C2}\cdot\overline{C1}\cdot\overline{C0}+\cdot\overline{C2}\cdot C1\cdot C0\]
	\[HEX1=\overline{C2}\cdot\overline{C1}\cdot\overline{C0}+\cdot\overline{C2}\cdot C1\cdot C0\]
	\[HEX0=\overline{C2}\cdot C0\]
	

	\begin{figure}[h]
		\centering
	 	\includegraphics[scale = 0.45]{Niki_puntoA/tabella.png}
	 	\caption{decoder truth table}
	\end{figure}	
	 
	 Finally the logic states are implemented using gates as shown in $figure 3$.\newline
	 The circuit is then described into VHDL using a dataflow style approach, the VHDL file is called $puntoA.vhd$. 
	 
	 
	 
	  \begin{figure}[!h]
	  \centering
	  \includegraphics[scale = 0.65]{Niki_puntoA/schema.png}
	  \caption{decoder gates implementation}
  	\end{figure}		
  
  The VHDL entry has been finally simulated via  $testbench approach$ where every possible input combination has been considered validating the output.
  The testbench results are shown below in figure 4.
  
  \begin{figure}[!h]
  	\centering
  	\includegraphics[scale = 0.65]{Niki_puntoA/Capturef.png}
  	\caption{Testbench results}
  \end{figure}





\section{ Multiplexing the 7-segments display output}
\section{ Binary to Decimal converter}
\section{ Binary-to-BCD Converter}




\end{document}
