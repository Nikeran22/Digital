\documentclass[12pt]{article}
\usepackage[english]{babel}
\usepackage{natbib}
\usepackage{url}
\usepackage[utf8x]{inputenc}
\usepackage{amsmath}
\usepackage{graphicx}
\graphicspath{{images/}}
\usepackage{parskip}
\usepackage{fancyhdr}
\usepackage{vmargin}
\usepackage{xcolor}
\usepackage{siunitx}
\usepackage{physics}
\setmarginsrb{3 cm}{2 cm}{3 cm}{2 cm}{1 cm}{1.5 cm}{1 cm}{1.5 cm}

\title{Lab 03}													% Title
\author{G 03}														% Author
\date{2 Apr 2019}														% Date

\makeatletter
\let\thetitle\@title
\let\theauthor\@author
\let\thedate\@date
\makeatother

\pagestyle{fancy}
\fancyhf{}
\rhead{\theauthor}
\lhead{\thetitle}
\cfoot{\thepage}
\newcommand{\mis}[3]{(#1 \pm #2) \ #3}
\newcommand{\misp}[3]{(#1 \#3 \pm #2}
\begin{document}

%%%%%%%%%%%%%%%%%%%%%%%%%%%%%%%%%%%%%%%%%%%%%%%%%%%%%%%%%%%%%%%%%%%%%%%%%%%%%%%%%%%%%%%%%

\begin{titlepage}
	\centering
    \vspace*{0.5 cm}
    \includegraphics[scale = 0.75]{polito.jpg}\\[1.0 cm]				% University Logo
    \textsc{\LARGE Politecnico di Torino}\\[2.0 cm]						% University Name
	\textsc{\Large Digital systems electronics\\ A.A. 2018/2019}\\[0.5 cm]		% Course Code
	\textsc{\Large Prof. G. Masera}\\[0.5 cm]		% Nome del Professore
	\rule{\linewidth}{0.2 mm} \\[0.4 cm]
	{ \huge \bfseries \thetitle \\ \small \thedate}\\
	\rule{\linewidth}{0.2 mm} \\[1.5 cm]
	
	\begin{minipage}{0.4\textwidth}
		\begin{flushleft} \large
			Berchialla Luca\\												%Cognomi e nomi
			Laurasi Gjergji
			\\
			
			Mattei Andrea\\
            Lombardo Domenico Maria\\
            
			\end{flushleft}
			\end{minipage}~
			\begin{minipage}{0.4\textwidth}
            
			\begin{flushright} \large
			236032\\													%Matricole
			238259\\
            233755\\
            233959\\
            
		\end{flushright}
        
	\end{minipage}\\[2 cm]
	
\end{titlepage}

%%%%%%%%%%%%%%%%%%%%%%%%%%%%%%%%%%%%%%%%%%%%%%%%%%%%%%%%%%%%%%%%%%%%%%%%%%%%%%%%%%%%%%%%%

\section{4-bit Sequential RCA}
\subsection{Implementation}

\begin{figure}[h]
	\centering
	\includegraphics[scale = 0.8]{immagini/B1.jpg}
	\caption{Top level entity}
\end{figure}


Here we had to implement a 4-bit Sequential Ripple Carry Adder. To build the circuit requested in $figure 1$ several sub-circuits have been implemented.\\
As first point a Full Adder was implemented as shown in $figure$ $2a$. The 4-bit adder was built using four full adders in a ripple carry architecture. Its overflow signal is generated by a $xor$ gate whose inputs are the $carry_{out}$ signals of the last two full adders.\\

\begin{figure}[h]
	\centering
	\includegraphics[scale = 0.4]{immagini/B2.jpg}
	\caption{Full Adder}
\end{figure}

The Register and the Flip-Flop were implemented using the code provided in the instructions.\\
A new 7-segments display decoder was needed in order to display properly 2's complement numbers. It uses two 7-segments displays to show respectively the sign and the magnitude of the number. The implementation was done by means of the $when-else$ statement following the truth table in $figure$ $3$.
\begin{figure}[h]
	\centering
	\includegraphics[scale = 0.7]{immagini/B3.jpg}
	\caption{2's complement 7-segments decoder}
\end{figure}\\
The top level entity that implements the circuit is in the file $lab3\_es1.vhd$ where all the components needed are instantiated and connected together.\\ As required, the the inputs $A$ and $B$ have been assigned to $SW_{3-0}$ and $SW_{7-4}$ respectively, $KEY_0$ to the negated asynchronous reset input and $KEY_1$ to the clock. The magnitude and the sign of $A$ are displayed in $HEX4$ and $HEX5$ respectively. The same is done for $B$ and $S$ in $HEX2,HEX3$ and $HEX0,HEX1$ respectively. The adder's overflow is shown by means of the red $LEDR_9$.\\

\subsection{Testbench}
To validate the correct behavior of the circuit a testbench was created. The $clock$ signal was created using a process and its period has been fixed to $10ps$, the reset signal instead has a period of $115ps$, not a multiple of the clock, to verify that is asynchronous. It  is active for $10ps$ and is generated just 10 times in order to have cleaner waveforms.\\
To test every possible combination of inputs it has been used a 8-bit counter which increments its value every $100ps$, the least significant 4 bits have been assigned to $A$ and the others to $B$.\\
Since the the values of $A,B$ and $S$ coded for the 7-segments display are the output of the circuit in the testbench are present several if clauses that  translate the output values to the 2's complement binary value of $A,B$ and $S$.\\
\\
\subsection{Timing analysis}
The maximum operating frequency of the circuit $f_{max}$ has been determined with the help of Quartus Prime and TimeQuest and is $f_{max}=650MHz$.\\
The longest path in the circuit in terms of delay is the one that starts from the MSB of a register where $A$ or $B$ are stored and arrives to the Flip Flop that stores the overflow signal. This path is longer than the other possible path that starts from one of the previous registers and arrives to the register where $S$ is stored, because the overflow signal is calculated by a $xor$ gate whose inputs are the $carry_{out}$ signals of the last two full adders. Therefore, with respect to the MSB of $S$, which goes directly to the register, the signal has to be processed by the $xor$ gate before arriving to the flip flop.


\section{4-bit Sequential Adder/Subtractor}
\subsection{Implementation}
The circuit implemented in this section is a little modification of the circuit implemented in section 1. In particular the  4-bit full adder has been modified to make it perform also the subtraction. The $carry_{in}$ input is replaced by the $add_subtract$ input  that enables the sum when its value is $0$ and the subtraction when is $1$. The $B$ input is 2's complemented when $add_subtract=1$ by assigning the $B_i$ input of each full adder to the xor of $B_i$ and $add_subtract$ and by assigning the $carry_{in}$ of the first full adder to $add_subtract$.\\ 
\subsection{Testbench}
The $add_subtract$ input was then assigned to $SW_8$ in the top level entity. In the testbench the counter has been modified now being 9-bit wide and its MSB is assigned to $add_subtract$.\\ 
\subsection{Timing analysis}
The maximum operating frequency of the circuit $f_{max}$ has been determined  as in section 1 and its value is $f_{max}=600MHz$. $f_{max}$ is lower for this circuit because the longest path now includes another level of logic that  is between the output of the register where $B$ is stored and the input of the full adder. \\
Therefore the longest path is the same as the previous section but it starts only from the previously mentioned register because only between  it and the full adder there are the additional $xor$ gates.\\


\section{3}
punto 3


\section{4}
punto 4

\end{document}
